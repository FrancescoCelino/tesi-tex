% !TEX TS-program = pdflatex
% !TEX encoding = IsoLatin

\documentclass[a4paper,11pt,twoside,english]{toptesi}
\usepackage[latin1]{inputenc}
\usepackage[T1]{fontenc}
\usepackage{lmodern}
\usepackage[a-1b]{pdfx}
\usepackage{hyperref}
\usepackage{cite}
\usepackage{listings}
\usepackage{color}
\usepackage{url}
\usepackage{multirow}
\usepackage{amsmath}
\usepackage{verbatim}
\usepackage{amssymb}


% Definition of a new language for listings
%\lstdefinelanguage{P4}
%{
%	% list of keywords
%	alsoletter=\#,
%	morekeywords={
%		header\_type,
%		fields,
%		header,
%		parser,
%		extract,
%		return,
%		default,
%		next,
%		\#define,
%		latest,
%		select,
%		apply,
%		control,
%		table,
%		reads,
%		actions,
%		modify\_field,
%		action,
%		valid,
%		nop,
%		exact,
%		lpm,
%		if,
%		hit,
%		and,
%		extern,
%		void,
%		bit
%	},
%	sensitive=true, % keywords are not case-sensitive
%	morecomment=[l]{//}, % l is for line comment
%	morecomment=[s]{/*}{*/}, % s is for start and end delimiter
%	morestring=[b]" % defines that strings are enclosed in double quotes
%}
%
%\definecolor{eclipseBlue}{RGB}{42,0.0,255}
%\definecolor{eclipseGreen}{RGB}{63,127,95}
%\definecolor{eclipsePurple}{RGB}{127,0,85}
%
%\lstset{
%	language={P4},
%	basicstyle=\fontsize{8}{9}\ttfamily, % Global Code Style
%	captionpos=b, % Position of the Caption (t for top, b for bottom)
%	extendedchars=true, % Allows 256 instead of 128 ASCII characters
%	tabsize=2, % number of spaces indented when discovering a tab 
%	columns=fixed, % make all characters equal width
%	keepspaces=true, % does not ignore spaces to fit width, convert tabs to spaces
%	showstringspaces=false, % lets spaces in strings appear as real spaces
%	breaklines=true, % wrap lines if they don't fit
%	frame=trbl, % draw a frame at the top, right, left and bottom of the listing
%	%frameround=tttt, % make the frame round at all four corners
%	framesep=4pt, % quarter circle size of the round corners
%	numbers=left, % show line numbers at the left
%	numberstyle=\tiny\ttfamily, % style of the line numbers
%	commentstyle=\color{eclipseGreen}, % style of comments
%	keywordstyle=\color{eclipsePurple}, % style of keywords
%	stringstyle=\color{eclipseBlue}, % style of strings
%}

\hypersetup{%
    pdfpagemode={UseOutlines},
    bookmarksopen,
    pdfstartview={FitH},
    colorlinks,
    linkcolor={blue},
    citecolor={blue},
    urlcolor={blue}
  }

\newtheorem{osservazione}{Osservazione}% Standard LaTeX
\newtheorem{definition}{Definition}


\begin{document}
\english

\ateneo{Politecnico di Torino}
\corsodilaurea{Mathematical Engineering}

\titolo{Modeling the new flu wave using data science and complex networks theory.}

\candidato{Francesco \textsc{Celino}}
\relatore{Prof.\ Lorenzo \textsc{Zino}}
\secondorelatore{Prof.\ Alessandro \textsc{Rizzo}}
\sedutadilaurea{\textsc{Academic~Year} 2024-2025}
\logosede{Cover/polito_new-1}


\CorsoDiLaureaIn{Master Degree course in\space}
\TesiDiLaurea{Master's Degree Thesis}
\InName{in}
\CandidateName{Candidate}
\AdvisorName{Supervisors}


\frontespizio % make cover page

\ringraziamenti

Ancora da fare...

\abstract
This Master's thesis explores how the SEINR (Susceptible, Exposed, Infectious, Non-infectious, Recovered) compartmental model can be used to forecast the evolution of influenza-like illness (ILI) cases in Italy during the 2023-24 winter season. The model's innovative aspects lie in its community-based meta-population framework, which simulates \textit{intra-} and \textit{inter-}regional mobility, capturing network dynamics critical to understanding how a disease spreads in Italy's diverse demographic and geographic landscape. This approach, which was previously successful when evaluating the efficacy of NPIs (non-pharmaceutical interventions) during the \textit{COVID-19} pandemic, is then further refined by taking into account features such as class divisions according to age, activity levels, and vulnerability to disease of different age groups, increasing the adaptability of the model to the Italian landscape.


Bayesian inference tools and Monte Carlo methods are then used to improve the estimates of a few key epidemiological parameters such as transmission rate and infection duration. Our empirical results demonstrate the model's effectiveness in capturing flu trends in Italy.


This work emphasizes the role of adaptive modeling in epidemiology, and how public health strategies driven by past data can help in managing seasonal epidemics. 

\indici

\mainmatter

\chapter{Introduction}
Each year, Influenza affects millions of people across the world, posing great challenges for healthcare systems in various developed countries, with great risks for elderly and more vulnerable people. In response, researchers and public health officials work to predict and control the spread of these diseases, using various tools and methods. One of the most powerful tools we have available is mathematical modeling, which allows us to simulate how a disease might spread through a population through the use of "differential equations". This thesis focuses on adopting one of such models\cite{par21} for flu-like illnesses in Italy, using a framework called the SEINR model.


The SEINR model is a type of "compartment" model. This means that it divides the population into different groups (or compartments) of people, based on the stages of the disease and whether or not they are infected. In this case, the compartments are \textit{Susceptible} (people who can catch the flu), \textit{Exposed} (people who have been infected but are not yet contagious), \textit{Infectious} (people who can spread the flu), \textit{Non-infectious} (people who do not present any symptom, or are isolated), and \textit{Removed} (people who are either immune to the flu after recovering, or dead due to complications). 


The history of compartmental models dates back to 1927, when Kermack and McKendrick\cite{ker27} defined the SIR model for the first time, in its simplest form. 


However, what makes the model we used for this thesis particularly innovative is its "meta-population" structure, already used with great success in 2021 when trying to model the first Covid-19 wave in Italy \cite{par21}. Instead of treating Italy as one large group of people, the meta-population approach breaks it down into regions and even considers how people move between these regions. This is crucial for a country like Italy, where people frequently travel between cities and regions for work, school, and other reasons. 


Most importantly, when trying to predict the outcomes of a Influenza season, we need to consider other crucial factors, such as vaccines, the age and activity levels of individuals, as well as their vulnerability to illness. These factors are important because some groups, like the elderly or those with pre-existing health conditions, are more at risk during a flu outbreak, but could very well be less exposed to the disease due to having less social contacts. If we want our predictions to be accurate, we need to take into account all of those factors. 


Another key feature of this thesis is the use of "Bayesian inference", a statistical method that allows the model to improve its accuracy (and precision) over time. As new weekly data about flu cases becomes available, the model updates its parameters in a probabilistic fashion: that means, if in one particular week we observe more infected people than we would expect, we probably need to rethink our predictions. That's why, when using Bayesian inference, we do not provide exact estimates for the number of infected people we will have next week, but rather a "probability distribution" that includes reasonable confidence intervals: as more weeks pass and we gather more real data, we expect those probability distributions to get narrower, reflecting our increased confidence on the disease's characteristics. 


In short, the goal of my thesis is to provide a more accurate and flexible tool for predicting flu-like illnesses in Italy, one that can adapt to different variants of Influenza each year (as you probably know, each year the flu outbreak is slightly different due to changing vaccines, mobility patterns, timing and public awareness). This tool can help public health officials better prepare for and respond to outbreaks, reducing the strain on healthcare systems and, who knows, maybe also save some lives. 


I am also proud to say that, thanks to the invaluable guidance of my supervisors, we were able to use this model to contribute to two major forecasting projects at the European level. Through these projects, I had the opportunity to collaborate with the \textit{ISI Foundation} and the \textit{Istituto Superiore di Sanita'}, as we provided weekly estimates for the evolution of last year's flu season.

\cite{par21}

%\section{Plagiarism}

%To be approved, your thesis must pass a plagiarism check. As a reminder:
%\begin{itemize}
%\item it is forbidden to copy verbatim one or more sentences  from any other external source (web pages, papers, etc.), unless you explicitly  grant the source and put the sentence/sentences in ``...''. Your can instead rephrase the whole sentence/sentences.
%\item it is forbidden to include an image (diagram or figure) taken from web pages or papers. It is better to redraw. If you really need to include  an image from an external source, you must add to the caption ``(Reproduced from~\cite{example})'', citing properly the source.
%\end{itemize}

 
\chapter{Methods}
The framework which will be used in this project is the one outlined by compartmental models, which aim to divide a population into distinct groups, or compartments, based on their status in relation to the disease. These models provide a simplified yet powerful way to describe the progression of an epidemic through a population using systems of ordinary differential equations (ODEs), in which each variable represents a compartment, and each parameter is modelled according to biological inferences. 

The earliest modern compartmental model is the SIR model, introduced by Kermack and McKendrick in 1927 \cite{ker27}. As implied by the name, the SIR model divides the population into three compartments: \textit{Susceptible} (S), representing individuals who can contract the disease; \textit{Infectious} (I), representing those actively spreading the disease; and \textit{Removed} (R), which includes individuals who have recovered and gained immunity or have died. The dynamics of the SIR model are governed by the following system of ODEs:

\begin{equation}
	\begin{aligned}
		\frac{dS}{dt} &= -\beta \frac{SI}{N}, \\
		\frac{dI}{dt} &= \beta \frac{SI}{N} - \gamma I, \\
		\frac{dR}{dt} &= \gamma I,
	\end{aligned}
\end{equation}

where \(N\) is the total population size, \(\beta\) is the transmission rate, and \(\gamma\) is the recovery rate. It is apparent that one of the key hypotheses of the SIR model is that a person cannot become infectious twice: once that person has recovered, they cannot become susceptible again. The model also assumes that our total population remains constant over time: the total sum of $ S + I + R $ cannot change over time, and this fact is easily verifiable by integrating these equations with respect to time. 

This model's simplicity, while being its greatest strength, is also a major weakness: the model does not contemplate an "asymptomatic period" of sorts, since as soon as you are taken out of the susceptible compartment you can already spread the disease to other people. This behaviour does not describe real-life epidemic phenomena correctly, but makes the model simpler conceptually and easier-to-use. 

For my thesis, a more complex model was needed: building upon the SIR framework, the \textbf{SEINR model} introduces additional compartments to better capture the complexities of real-world epidemics. Specifically, the SEINR model includes:

\begin{itemize}
	\item \textbf{Susceptible (S):} Individuals who can contract the disease.
	\item \textbf{Exposed (E):} Individuals who have been infected but are not yet infectious, representing the incubation period.
	\item \textbf{Infectious (I):} Individuals who can transmit the disease.
	\item \textbf{Non-infectious (N):} Individuals who no longer spread the disease, either because they are isolated, mildly ill, or simply no longer symptomatic. 
	\item \textbf{Removed (R):} Individuals who have either recovered and gained immunity or succumbed to the disease.
\end{itemize}

The inclusion of the \textbf{Exposed} and \textbf{Non-infectious} compartments allows the SEINR model to more accurately reflect diseases with an incubation period or asymptomatic cases, which are common in influenza-like illnesses. The same hypotheses we made with the SIR model about the conservation of the population (no births or deaths unrelated to the disease) and the impossibility of becoming ill twice apply here. 

An important concept that arises in compartmental models with a \textbf{Removed} compartment is \textbf{herd immunity}. Herd immunity occurs when a sufficient portion of the population becomes immune, either through infection or vaccination, thereby reducing the probability of disease transmission to susceptible individuals. This phenomenon is closely related to the \textbf{basic reproduction number} (\(R_0\)), a key epidemiological parameter that represents the average number of secondary infections generated by a single infectious individual in a fully susceptible population. In other words, if a disease has a basic reproduction number greater than one, that disease should theoretically spread indefinitely, since each person infects more than one person before recovering (on average). 

For simplicity's sake, let us consider the SIR model, where herd immunity is achieved when the fraction of the population that remains susceptible falls below a critical threshold, given by:

\begin{equation}
	S_c = \frac{1}{R_0}.
\end{equation}

At this point, the effective reproduction number (\(R_t = R_0 \cdot \frac{S}{N}\)) drops below 1, causing the epidemic to decline. This principle also holds true for more complex models like SEINR, where, as explained, the dynamics of immunity and disease spread are also influenced by latency periods and non-infectious stages. This behaviour is harder to describe in analitical terms, but will be numerically visible in the following experiments. 

In the context of seasonal influenza, if we make the assumption that a person can only become infected once each year, the concept of herd immunity plays a significant role in shaping public health strategies, particularly in designing vaccination campaigns (many compartmental models automatically include vaccinated people in the removed compartment). By estimating \(R_0\) and tracking its progression as the epidemic goes on, one can gain insight into how contagious a certain virus or bacteria strain is in a certain place at a certain time, and thus whether implementing non-pharmaceutical interventions (NPIs), such as social distancing or lockdowns, is warranted. 

%Inizia parlando dei modelli a compartimenti in generale, introduci il SIR, parla della struttura a equazioni diff. alle derivate ordinarie, introduci il modello SEINR generale coi 5 diversi compartimenti. Parla del concetto di immunit� di gregge nei modelli col compartimento removed, in funzione del numero di riproduzione di base
 
\section{The Metapopulation SEINR Framework}

\subsection{Activity-Driven Meta-Population Model}

In the \textit{Metapopulation} framework, we partition a population of \(n\) individuals into \(K\) communities, denoted as \(\mathcal{H} = \{1, \ldots, K\}\): each community represents bounded and well-defined geographical areas (e.g., regions, provinces, or cities). Each community \(h \in \mathcal{H}\) contains \(n_h\) people. The way a community interacts with other communities in the model is described by a \textit{weighted graph}, where each edge represents a travel path. The weights of this graph are described using the \textit{routing matrix} \(\mathbf{W} \in [0,1]^{K \times K}\), which defines the fraction of individuals in a certain community that move to other communities when becoming "active". For example, \(\mathbf{W}_{hk}\) denotes the fraction of individuals from community \(h\) that travel to community \(k\) in a given time unit (as we will see, a time unit represents a day in our model). The matrix satisfies:
\[
\mathbf{W}_{hh} = 0, \quad \sum_{k=1}^K \mathbf{W}_{hk} = 1, \quad \forall h.
\]

The inhabitants of each community are then divided into \(P\) "activity classes", \(a_1, a_2, \ldots, a_P\), where \(0 < a_i \leq 1\). The baseline activity \(a_i\) quantifies the propensity of individuals in class \(i\) to interact with other people. At each time step, a fraction \(a_i\) of individuals in each class becomes active, either interacting "locally" or traveling to other communities as determined by the \textit{mobility parameter} \(b \in [0,1]\). This parameter simply represents the fraction of active individuals commuting to other communities (thus coming into contact with people from other communities), with the remaining \(1 - b\) interacting within their own community.

\subsection{Disease Progression}

As explained, the dynamics of the disease will be modeled using the susceptible--exposed--infectious--non-infectious--removed (SEINR) framework. After contagion, susceptible individuals move into the \textit{Exposed} (\(E\)) compartment with rate \(\lambda\), representing the latency period before becoming infectious. The transitions between compartments are defined as follows:
\begin{itemize}
	\item \(E \to I\): Transition to the \textit{Infectious} (\(I\)) compartment occurs at rate \(\nu\), with \(1/\nu\) representing the average latency period.
	\item \(I \to N\): Transition to the \textit{Non-infectious} (\(N\)) compartment occurs at rate \(\mu\), where \(1/\mu\) is the average infectious period.
	\item \(N \to R\): Transition to the \textit{Removed} (\(R\)) compartment occurs at rate \(\gamma\), while \(1/\gamma\) represents the average delay before recovery or death.
\end{itemize}
It is clear that the average time from infectiousness to removal is \(1/\mu + 1/\gamma\). The transition between compartments is regulated by the following set of ODEs:
\[
\begin{aligned}
	S_i^h(t+1) &= \big(1 - \Pi_i^h(t)\big) S_i^h(t), \\
	E_i^h(t+1) &= \Pi_i^h(t) S_i^h(t) + \big(1 - \nu\big) E_i^h(t), \\
	I_i^h(t+1) &= \nu E_i^h(t) + \big(1 - \mu\big) I_i^h(t), \\
	N_i^h(t+1) &= \mu I_i^h(t) + \big(1 - \gamma\big) N_i^h(t).
\end{aligned}
\]

Where \(\Pi_i^h(t)\) represents the contagion probability, which will be defined in the following section. 
\subsection{Contagion Mechanism and Contact Probabilities}

If an infectious person comes into contact with a susceptible person, the latter will not always become infected. This uncertainty is captured by the probability of contagion, \(\Pi_i^h(t)\), represents the fraction of susceptible individuals in activity class \(i\) in community \(h\) that transition to the \(E\) compartment due to an interaction with infectious individuals. Since we are dealing with large populations, we can assume a thermodynamic limit (\(n \to \infty\)) and low epidemic prevalence (as we will see, this is in line with real-life data), and \(\Pi_i^h(t)\) can be expressed as:
\[
\Pi_i^h(t) = m \alpha a_i (1 - \beta b) \lambda P_h + m (1 - \alpha \beta a_i b) \lambda Q_h + m \alpha \beta a_i b \sum_{k \in \mathcal{H}} \mathbf{W}_{hk} \lambda P_k + m \alpha \beta a_i b \sum_{k \in \mathcal{H}} \mathbf{W}_{hk} \lambda Q_k,
\]
where:
\[
P_h = \frac{1}{\tilde{n}_h} \left( \sum_{j=1}^P (1 - \alpha \beta a_j b) I_j^h + \sum_{k \in \mathcal{H}} \mathbf{W}_{hk} \sum_{j=1}^P \alpha \beta a_j b I_j^k \right),
\]
\[
Q_h = \frac{1}{\tilde{n}_h} \left( \sum_{j=1}^P (1 - \beta b) \alpha a_j I_j^k + \sum_{k \in \mathcal{H}} \mathbf{W}_{hk} \sum_{j=1}^P \alpha \beta a_j b I_j^k \right),
\]
and \(\tilde{n}_h\), is the effective population size in community \(h\). This is defined as

Thus, we now have a SEINR framework that incorporates interactions between various communities.

%Spiega cosa c'e' di nuovo nell'approccio a meta-popolazione, come funzionano i contatti fra individui, come ci si sposta. Spiega cos'e' una matrice di pendolarismo, e come si calcolano le probabilit� di contatto. 

\subsection{Age classes}
Some age brackets, such as the elderly, tend to have fewer social interactions compared to younger people. However, they also have a significantly higher probability of developing severe complications in case of infection. 

To account for this, the population was partitioned in 2 age classes, the former of which contains every individual who is less than 65 years old, and the latter includes everyone else. The fraction of individuals in each age classes was calibrated using italian census data\cite{istat}. 

\subsection{Commuting matrix}
The commuting matrix is estimated using official census data\cite{istat}. Each entry \(W_{hk}\) of the matrix denotes the proportion of individuals from region \(h\) that commute to region \(k\). The structure of the commuting matrix significantly influences disease spread since regions with high incoming mobility may experience faster outbreaks due to external infections. 

Figure~\ref{fig:commuting_matrix} illustrates the structure of the commuting matrix used in our model.

\begin{figure}[h]
	\centering
	\includegraphics[width=0.7\textwidth]{Figures/commuting_matrix.png}  % Replace with actual figure file
	\caption{Representation of the commuting matrix \(W\) showing mobility patterns between different regions. Taken from \cite{par21}}
	\label{fig:commuting_matrix}
\end{figure}

\subsection{Vaccines}

In Italy, about two-thirds of the elderly population becomes vaccinated against Influenza each year, with the vaccination campaign typically occurring between October and December \cite{vaxita23}. 

To integrate the concept of vaccination in our model, we created a new class for \textit{vaccinated elderly} people (with its own activity level $a_elderly$), in addition to the existing classes for the general elderly and young populations. The transition from the \textit{elderly} class to the \textit{vaccinated elderly} class occurs instantaneously once a predefined threshold date is reached, as an approximation of the timing of the real-world vaccination campaign.

Vaccine efficacy values are taken from \cite{vax23} and are applied to reduce the probability of transition from the \textit{susceptible} to the \textit{exposed} compartment. This adjustment reflects the protective effect of vaccination against infection. In addition to that, vaccinated individuals retain a lower probability of developing severe complications, in line with empirical data on vaccine effectiveness.
 
\section{Other parameters}
\begin{table}[h]
	\centering
	\renewcommand{\arraystretch}{1.2}
	\begin{tabular}{|l|c|l|}
		\hline
		\textbf{Meaning} & \textbf{Value(s)} & \textbf{Reference} \\
		\hline
		$1/\nu$ & Latency period & \\
		$1/\mu$ & Infectiousness period & \\
		$1/\gamma$ & Time from infectiousness to reported death & \\
		$\lambda$ & Per-contact infection probability & $\surd$ \\ 
		$\eta$ & Class distribution & \\
		$a$ & Baseline activity & \\
		$b$ & Mobility parameter & \\
		$\alpha_{\text{low}}$ & Activity reduction & $\surd$ \\
		$m$ & Average number of contacts & \\
		$\beta_{\text{low}}$ & Mobility reduction & $\surd$ \\
		\hline
	\end{tabular}
	\caption{Summary of additional model parameters.}
	\label{tab:parameters}
\end{table}


\section{Real-world Data}

In order to validate our model, it is necessary to rely on real-world epidemiological data: in our case, we decided to utilize the weekly incidence of influenza-like illnesses (ILI) at the regional level. These data are sourced from the official \textit{Influcast} project repository on GitHub \cite{github}, which provides up-to-date weekly reports on influenza incidence across Italian regions. These data are publicly available, and the information provided has its roots in a network of Italian doctors that decided to contribute to the \textit{Influcast} project by sending data about how many of their patients show signs of flu illness. 

\subsection{Data Format}

The data are provided as CSV files, where each row represents the recorded incidence for a specific week in a given region. Each file (for example \textit{marche-2023-52-ILI.csv} has the following format):
\begin{itemize}
	\item \textbf{Year} (\texttt{anno}): The calendar year of the observation.
	\item \textbf{Week} (\texttt{settimana}): The calendar week number.
	\item \textbf{Number of Cases} (\texttt{numero\_casi}): The number of reported ILI cases in the region by the surveillance system.
	\item \textbf{Number of Assisted Individuals} (\texttt{numero\_assistiti}): The total number of patients monitored by the surveillance system.
	\item \textbf{Incidence} (\texttt{incidenza}): The estimated incidence per 1000 inhabitants (in our case, this is simply the number of cases divided by the total number of monitored people times 1000).
	\item \textbf{Target} (\texttt{target}): The type of disease being monitored (ILI, in this case).
\end{itemize}

Table~\ref{tab:csv_format} shows an example of this structure:

\begin{table}[h]
	\centering
	\renewcommand{\arraystretch}{1.2}
	\begin{tabular}{|c|c|c|c|c|c|}
		\hline
		\textbf{Year} & \textbf{Week} & \textbf{Cases} & \textbf{Assisted} & \textbf{Incidence} & \textbf{Target} \\
		\hline
		2023 & 45 & 137.0 & 24447.0 & 5.6 & ILI \\
		2023 & 46 & 197.0 & 29651.0 & 6.64 & ILI \\
		2023 & 47 & 216.0 & 33744.0 & 6.4 & ILI \\
		2023 & 48 & 335.0 & 36037.0 & 9.3 & ILI \\
		2023 & 49 & 349.0 & 30838.0 & 11.32 & ILI \\
		2023 & 50 & 524.0 & 29366.0 & 17.84 & ILI \\
		\hline
	\end{tabular}
	\caption{Example of CSV file structure for weekly influenza incidence data.}
	\label{tab:csv_format}
\end{table}

The reason why we decided to rely on real-world data for the calibration of our model is two-fold:
\begin{itemize}
	\item **Initial Conditions**: Each system of ODEs requires a few initial conditions in order to be simulated. The best thing to do in this case is to provide an imput that makes sense, in order to get an output that can also be applied to the real world. 
	\item **Model Calibration and Training**: during the training phase of our model, we can penalize "bad predictions" and assign good scores to "good predictions" by comparing the output of our model with what we can see in the real world, and adjust the parameters of our model accordingly.
\end{itemize}


\subsection{Data Manipulation}

A key challenge in integrating real-world data into epidemiological models is interpretation, and the difference in spatial resolution: while the incidence data are available at the \textbf{regional level}, our model operates on a \textbf{provincial scale}. To bridge this gap, we distribute the reported regional incidence among the provinces proportionally to their respective populations.

Mathematically, given a region \( r \) with a total population \( P_r \) and an incidence rate \( I_r \), the estimated number of infected individuals in province \( p \) (within region \( r \)) is computed as:
\[
I_p = I_r \cdot \frac{P_p}{P_r},
\]
where \( P_p \) is the population of province \( p \).

While this proportional allocation is a reasonable first approximation, it introduces a couple of limitations:
\begin{itemize}
	\item The incidence rate does not necessarily scale linearly with population. Larger cities may experience higher transmission rates due to higher population density and mobility.
	\item Urban centers may tend to act as hubs for disease spread, meaning that real incidence values may be skewed compared to our proportional model.
\end{itemize}
Thus, the true relationship between incidence and population might be better captured by a nonlinear function (e.g., polynomial or exponential), an approach that our current model does not consider.

Despite these limitations, our approach ensures that the initial conditions used by the model are at least somewhat demographically consistent with observed epidemiological data. Future improvements may involve incorporating mobility data or using historical patterns of disease spread to refine the distribution of incidence values from the regional level to the provincial level.


\subsection{Model Calibration}

We estimated the initial values of a few epidemiological parameters using virological studies on influenza \cite{flu07, flu21}. These estimates gave us a biologically plausible range for each parameter. We then refined these parameters through a fine-tuning process, utilizing historical data on flu outbreaks of past years. 

\subsubsection*{Initial Estimation from Virological Studies}

We initially estimated the order of magnitude of the following parameters based on virological literature:
\begin{itemize}
	\item \(\mu\): The rate at which individuals transition from the \textit{Exposed} (\(E\)) compartment to the \textit{Infectious} (\(I\)) compartment (i.e., the inverse of the latency period).
	\item \(\beta\): The rate at which individuals transition from the \textit{Infectious} (\(I\)) compartment to the \textit{Non-infectious} (\(N\)) compartment (i.e., the inverse of the infectious period).
	\item \(\lambda\): The rate of transmission, governing the transition from \textit{Susceptible} (\(S\)) to \textit{Exposed} (\(E\)).
	\item \(\gamma\): The rate at which individuals transition from the \textit{Non-infectious} (\(N\)) to the \textit{Removed} (\(R\)) compartment (i.e., the inverse of the recovery/removal period).
\end{itemize}

Table~\ref{tab:flu_parameters} provides a few examples of typical values for these parameters based on the current literature.

\begin{table}[h]
	\centering
	\renewcommand{\arraystretch}{1.2}
	\begin{tabular}{|c|c|}
		\hline
		\textbf{Parameter} & \textbf{Estimated Range (Influenza)} \\
		\hline
		\(1/\mu\) (Latency Period) & 1.5-2 days \\
		\(1/\beta\) (Infectious Period) & 3-5 days \\
		\(1/\gamma\) (Recovery/Removal Time) & 5-7 days \\
		\(\lambda\) (Per-Contact rate of transmission) & Estimated numerically \\
		\hline
	\end{tabular}
	\caption{Estimated parameter ranges for influenza based on virological studies \cite{flu07, flu21}.}
	\label{tab:flu_parameters}
\end{table}

\subsubsection*{Fine-Tuning with Historical Data}

After defining reasonable initial estimates, we refined these parameters using historical influenza incidence data from past seasons. The fine-tuning process involved adjusting \(\mu\), \(\beta\), and \(\lambda\) to minimize the discrepancy between the simulated epidemic curves and observed influenza incidence trends.

The optimization was performed iteratively by running the model with different parameter sets and comparing the output to real-world data. The primary metric used for assessing the quality of fit was the **Mean Absolute Error (MAE)** between the simulated and observed incidence data. Since influenza dynamics vary from year to year, calibration was performed separately for each season to account for changes in viral transmissibility, population immunity, and public health interventions.

By combining virological knowledge with empirical calibration, our model ensures both biological plausibility and high predictive accuracy when applied to real-world influenza outbreaks.


\section{A different approach: Bayesian estimates}
parla della seconda parte del lavoro, cioe' riscrivere il modello per ottenere un aggiornamento in tempo reale dei parametri scelti, man manoche uscivano i nuovi dati reali settimanali sull'incidenza.  

\subsection{Theory of Bayesian Estimates}
Parla molto brevemente del teorema di Bayes, che cos'e' una prior, una posterior e la likelyhood, e di come si possono usare questi concetti nel nostro modello, creando delle distribuzioni a priori per i parametri lambda, nu e beta. Spiega le differenze con l'approccio frequentista in probabilita'. 

\subsection{Implementation}
Spiega come implementare questo approccio nel modello SEINR tramite Metodi Monte Carlo per via della forma complicatissima della likelyhood

\chapter{Coding and Implementation}
Parla delle librerie di python utilizzate, dell'approccio "discreto" usato per risolvere le eq. diff. ordinarie

\section{Coding Bayesian Inference}
Spiega la parte di codice relativa a Bayes

\section{Uploading our results}
Spiega finalmente come abbiamo formattato i dati per inviarli a Influcast e Respicast, metti qualche tabella
\section{Related work}\label{sec:rel}
Menzionare brevemente i lavori precedenti a cui ci siamo ispirati per il modello

\chapter{Experiments and Contributions}\label{sec:cont}
Spiega che esperimenti abbiamo fatto, le stime che abbiamo fornito e a quali progetti abbiamo lavorato
\section{Collaborative Projects}
\subsection{Influcast}
\subsection{Respicast}
\subsection{Collaborative paper published}
\chapter{Model Validation}
Confronta i risultati ottenuti con i due metodi con le varie metriche di errore (log score, weighted interval score, mean relative error, mean absolute error, coverage) spiegando pregi e difetti di ognuna

\section{Christmas Holiday issue}
Parla del picco di casi a Natale e mostra che l'utilizzo di Bayes risolve parzialmente il problema, mostra che tutti i modelli di Influcast hanno fatto fatica in quel periodo
\section{Methodology}
Spiega come verificare quale dei due approcci (bayesiano e puramente deterministico) si comporta meglio nel periodo critico natalizio

\section{Numerical results}

Metti tutti i grafici e tutte le tabelle per le varie regioni con le metriche dell'errore\cite{influcast}
%\begin{figure}[tb!]
%\begin{center}
%\includegraphics[width=10cm]{Figures/sino.pdf}
%\caption{grafico di esempio.}\label{fig:bw}
%\end{center}
%\end{figure}

%Some basic hints:
%\begin{itemize}
%\item both x and y axes should be properly labeled. %Units of measurements must be {\em always} reported!
%\item if the curve are experimental or by simulation, remember to report always all the points in addition to the line connecting them
%\end{itemize}




\chapter{Conclusion}

Riporta molto brevemente che risultati abbiamo ottenuto, ricalca l'importanza dell'approccio bayesiano. 

Spiega i punti deboli del nostro modello: la struttura provinciale della matrice di pendolarismo si sposa male col fatto che i dati reali sono su base regionale. Il fatto che i dati reali vengano forniti su base settimanale influisce negativamente, dato che il modello ragiona in maniera giornaliera. Servono anche piu' dati sull'efficacia del vaccino, e l'applicabilita' del nostro modello e' limitata a singole stagioni influenzali in cui non si prevede la possibilita' di reinfezione. 

L'approccio bayesiano probabilmente migliora la precisione del modello (cio' va confermato nei prossimi anni se vogliamo esserne sicuri) ma aumenta in maniera drammatica i costi computazionali, e se si vuole ovviare a questo problema e' necessario applicare l'inferenza bayesiana a pochissimi parametri. Menziona brevemente la questione underfitting e overfitting, e chiedersi se includere il compartimento N comporti dei guadagni in accuratezza. 

Proponi spunti per espandere il modello e implementare stime meno onerose per i parametri con il machine learning o l'intelligenza artificiale

\bibliographystyle{plain}
\bibliography{biblio}

\end{document}
